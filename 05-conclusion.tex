\chapter{結論}
\label{chap:conclusion}


本研究では一つのノードを占有して実行されるアプリケーションを想定した。ユーザにアプリケーションの異なる処理を行っている区間の区切りを示すコードを埋め込ませることで、アプリケーションを電力ー実行時間曲線が異なる複数の区間に分割した。そして、分割された区間の中で電力をかけても実行時間があまり短縮されない区間から電力をかけると実行時間が大きく短縮される区間へ、蓄電池を用いて電力が融通できるように充放電計画を立てることにより、電力制約を守りながら実行時間が短縮することを提案した。本提案手法の適用により、同程度の実行時間であるフェーズを複数持つようなCPU並列のアプリケーションでは平均4.5\%、GPU並列アプリケーションでは平均15\%程度の実行時間の短縮がなされることを確認した。

本提案手法では、電池容量・充放電速度が無限大の理想的な蓄電池を想定している。また、充放電計画の決定アルゴリズムとして全探索を用いており、上記の結果は理想的に電力融通がなされた場合の結果と言える。しかしアプリケーションを分割する部分では、第三者である著者がソースコードを読むことによってベンチマークアプリケーションの異なる処理部分を分割しているため、より良くアプリケーションの性質を理解しているアプリケーション作成者自身が分割した方が効果が高いと考えられる。そのため、フェーズ分割の精度向上によるさらなる性能向上の余地が残されている。

本手法を実際のシステムに適用するまでには、蓄電池の電池容量・充放電速度・最小放電間隔や寿命などの物理的制約を含めた、本手法よりも計算量の小さなアルゴリズムの構築が必要である。この実現のためには\ref{chap:experiment}章で定義したようなエネルギ実行時間短縮率などの評価関数を用いることが有用であると考えられる。