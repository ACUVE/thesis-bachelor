\chapter{結論}
\label{chap:conclusion}


本研究では、蓄電池を用いた、電力制約下における性能向上手法を提案した。
具体的には、まずユーザに処理区間の境界を示す指示文をソースコードに埋め込ませる。
次に、アプリケーションごとに電力ー実行時間曲線を取得し、
埋め込んだ情報を元にこの曲線を複数区間に分割する。
そして、電力制約下で実行時間を最小化する様に、区間ごとの蓄電池の充放電計画を立てる。
本提案手法の効果を実機上で検証した結果、実行時間の長さが同程度のフェーズを複数持つCPU用の並列アプリケーションでは平均4.5\%、GPU用の並列アプリケーションでは平均17.1\%の実行時間の短縮がなされることを確認した。


ただし、本提案手法では、電池容量・充放電速度が無限大の理想的な蓄電池を想定している。
現実の蓄電池では、電池容量・充放電速度・最小放電間隔や寿命などに物理的制約があるため、
これを考慮する必要がある。
また、本研究では充放電計画の決定アルゴリズムとして全探索を用いている。
しかし、上記の様に制約条件を増やすと計算量も増大するため、
全探索よりも計算量の小さなアルゴリズムの構築が必要となる。
そのためには、\ref{chap:experiment}章で定義したようなエネルギ実行時間短縮率などの評価関数を用いることが有用であると考えられる。


また、今後の更なる展望としてCPU・GPUの両者を用いて並列処理を行う場合への、本手法の応用が考えられる。
この場合、蓄電池を用いた時間方向の電力融通と、CPU・GPU間の空間方向の電力融通の相乗効果により、
更なる電力対性能の向上が期待できる。

