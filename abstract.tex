近年、コンピュータの消費電力の増大が大きな問題となっており、コンピュータの性能の指標として単なる実行速度だけではなく、消費電力あたりの実行速度(ワットパフォーマンス)が重要視されるようになってきている。特にスーパーコンピューターのような今日の高性能計算システムでは数メガワットもの電力を消費しており、物理的制約からこれ以上の電力供給力の向上は困難である。このような背景により、予め決められた消費電力の制約下での実行速度の最大化が、今後の高性能計算システムの性能向上の鍵となっている。

そこで、本論文では蓄電池を用いた高性能計算システムの性能を向上手法を提案する。現在の高性能計算システムには、停電時にもシステムへの電力供給を続けられるようにUPS(無停電電源装置)が搭載されている。それを非停電時にも積極的に充放電を行い、アプリケーションの中の電力対性能が上がりにくい部分から上がりやすい部分へ時間方向に電力を融通することによって、電力制約下における性能を向上させることができる。今回はこの手法をCPU-GPUハイブリッド構成の計算ノードを用いて3種類のベンチマークで性能評価実験を行い、この手法を用いない場合に比べて平均?\%の性能向上が実現できることを示し、その有用性を確認した。
