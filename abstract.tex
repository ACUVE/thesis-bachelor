近年、コンピュータの消費電力の増大が大きな問題となっている。そのためコンピュータの性能の指標として単なる実行速度だけではなく、消費電力あたりの実行速度が重要視されるようになってきている。特にスーパーコンピュータのような今日の高性能計算システムでは数メガワットもの電力を消費しており、物理的制約からこれ以上の電力供給力の向上は困難である。このような背景により、予め決められた消費電力の制約下での実行速度の最大化が、今後の高性能計算システムの性能向上の鍵となっている。

そこで、本論文では蓄電池を用いた高性能計算システムの電力制約下での性能向上手法を提案する。現在の高性能計算システムには、停電時にもシステムへの電力供給を続けられるようにUPS(無停電電源装置)が搭載されている。それを非停電時にも積極的に充放電を行い、アプリケーションの中の電力をかけても性能が上がりにくい部分から上がりやすい部分へ、電力を時間方向に融通することによって、電力制約下における性能を向上させることができる。本研究ではCPU-GPUハイブリッド構成の計算ノードを用いて、CPU上での並列アプリケーション及びGPU上での並列アプリケーションから2種類ずつのベンチマークを選び、本手法を用いた場合の性能評価実験を行った。その結果、本手法を上手く適用できるアプリケーションに関しては、本手法を用いない場合に比べてCPU上での並列アプリケーションでは平均4.5\%、GPU上での並列アプリケーションでは平均17.1\%の性能向上が実現できることを示し、その有用性を確認した。
