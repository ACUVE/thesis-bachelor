\chapter{研究の背景}
\label{chap:background}

本章ではまず提案手法の核となる技術であるDVFS、及びDVFSを用いた既存の電力削減手法について説明する。そして、対象とする蓄電池を含んだシステムの電力供給システムについて説明した後、蓄電池とDVFSの両方を用いた電力削減手法の関連研究を紹介する。

\section{DVFS}
\label{sec:dvfs}

\subsection{DVFSとは}

基本的に、プロセッサやメモリはある一定の周波数で動作するように設計されている。動作周波数が高いほど処理能力も高くなるが、同様に消費電力も大きくなる。かつてのプロセッサやメモリは設計時に決められた一つの動作周波数でしか処理を行うことはできなかったが、現在では一つのプロセッサやメモリが複数の動作周波数をサポートしており、演算中であっても瞬時に動作周波数を切り替えられるようになった。この技術を用いて動的に動作周波数を切り替えることによって、処理速度と消費電力を変化させる手法をDVFS(Dynamic Voltage and Frequency Scaling)と呼ぶ。


\subsection{DVFSを用いたコンピュータの既存の省電力化手法}

プログラム実行時、メモリやネットワークなどのプロセッサ以外のモジュールがボトルネックとなっているときには、プロセッサはビジーループとなり、処理を行わず電力だけを消費している時間の割合が高くなる。そのためこのような状況ではプロセッサ自体の処理能力を落としてもシステム全体の処理能力はあまり下がらないため、プロセッサを低い動作周波数に切り替えることで性能低下を防ぎつつ省電力化を行ってきた。

同様に、メモリがボトルネックとなっていない状態ではメモリの動作周波数を落とすことで電力を削減することができる\cite{David:2011:MPM:1998582.1998590}。

また、近年では複数のプロセッサを搭載したマルチプロセッサシステムが増えてきた。マルチプロセッサシステムは複数のプロセッサで並列に処理を行うことで高速化をはかっている。しかし、ひとつずつ順番に処理を行うことが必要なプログラムではひとつのプロセッサのみが処理を行っており、その他のプロセッサはほとんど処理を行っておらず、無駄な消費電力が発生していた。そのような状況では、処理を行っているひとつのプロセッサのみを高い周波数で動作させ、その他のプロセッサの動作周波数を落とすことで消費電力を削減している。

\subsection{プロセッサとメモリのDVFSと組み合わせた電力削減の関連研究}

一般に、プログラム実行時はプロセッサかメモリのどちらかの処理能力がシステム全体のボトルネックとなっていることが多く、このときボトルネックとなっていないモジュールでは処理能力が必要以上に高い状態となっており、電力が無駄に消費されている。そのため、それぞれのモジュール間での処理能力の差をなくすことが、無駄な電力消費を減らす上で重要である。

この問題を解決するため、プロセッサとメモリのDVFSを同時に用いることによって、それぞれのDVFSを別々に行う場合よりもさらに電力対性能の向上を目指した手法が存在する\cite{6493615}。この手法では、一定時間おきにプロセッサとメモリの処理能力の両方を監視して、一方のモジュールの処理能力が足りないときにはそのモジュールに電力を融通することによって処理能力の偏りをなくし、与えられた性能制約を満たしつつ省電力化を行っている。






\section{蓄電池を含む電力供給システム}
\label{sec:ups}


\section{データセンターにおける蓄電池を用いたピーク電力削減手法}
\label{sec:capping}

Power Cappingの先行研究~\cite{Fan:2007:PPW:1273440.1250665}.
