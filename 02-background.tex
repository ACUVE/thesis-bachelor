\chapter{研究の背景}
\label{chap:background}

本章ではまず提案手法の核となる技術であるDVFS、及びDVFSを用いた既存の電力削減手法について説明する。そして、対象とする蓄電池を含んだシステムの電力供給システムについて説明した後、蓄電池とDVFSの両方を用いた電力削減手法の関連研究を紹介する。

\section{DVFS}
\label{sec:dvfs}

\subsection{DVFSとは}

基本的に、プロセッサはある一定の周波数で動作するように設計されている。動作周波数が高いほど処理能力も高くなるが、同様に消費電力も大きくなる。かつてのプロセッサは設計時に決められた一つの動作周波数でしか処理を行うことはできなかったが、最近では一つのプロセッサが複数の動作周波数をサポートしており、演算中であっても瞬時に動作周波数を切り替えられるようになった。この技術を用いて動的に動作周波数を切り替えることによって、処理速度と消費電力を変化させる手法をDVFS(Dynamic Voltage and Frequency Scaling)と呼ぶ。


\subsection{DVFSを用いたコンピュータの既存の省電力化手法}

プログラム実行時、メモリやネットワークなどのプロセッサ以外のモジュールがボトルネックとなっているときには、プロセッサはビジーループとなり、処理を行わず電力だけを消費している時間の割合が高くなる。そのためこのような状況ではプロセッサ自体の処理能力を落としてもシステム全体の処理能力はあまり下がらないため、プロセッサを低い動作周波数に切り替えることで性能低下を防ぎつつ省電力化を行ってきた。

また、近年では複数のプロセッサが搭載されたシステムが増えてきた。マルチプロセッサシステムは複数のプロセッサで並列に処理を行うことで高速化をはかっている。しかし、ひとつずつ順番に処理を行うことが必要なプログラムではひとつのプロセッサのみが処理を行っており、その他のプロセッサはほとんど処理を行っておらず、無駄な消費電力が発生していた。そのような状況では、処理を行っているひとつのプロセッサのみを最高周波数で動作させ、その他のプロセッサの動作周波数を落とすことで消費電力を削減している。




\section{蓄電池を含む電力供給システム}
\label{sec:ups}


\section{データセンターにおける蓄電池を用いたピーク電力削減手法}
\label{sec:capping}

Power Cappingの先行研究~\cite{Fan:2007:PPW:1273440.1250665}.
