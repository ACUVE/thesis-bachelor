\chapter{序論}

現代社会においてコンピュータの担う役割はかつてないほど大きくなっており、
コンピュータシステムへのますますの高性能化が強く要求されている。
より高性能なコンピュータを作るべく、これまで多くの研究者がコンピュータ技術の発展に貢献し、
Mooreの法則\cite{mooreslaw}の示す通りチップの集積度が指数関数的に増すと共にコンピュータの性能も向上し続けてきた。



しかし近年では、消費電力の増大によってコンピュータの性能向上が妨げられてきている。
これは一般に、より高速な演算を行うためにはより大きな電力を消費しなければならず、性能向上と共に消費電力も増加し続けてきたためである。
その結果として、スーパーコンピュータなどの高性能計算機システムにおいては既に供給できる限界に近い電力を消費しており、
それによってコンピュータの性能が制限されている。
この様に、与えられた電力制約の中でいかに処理能力を向上させるかが現在の大きな課題となっている。


コンピュータの消費電力を削減するため、DVFS(Dynamic Voltage and Frequency Scaling)と呼ばれる技術が提案され、
現在では広く利用されている。DVFSとはプロセッサやメモリの動作速度を、負荷に応じて動的に変動させる技術である。
一般に、プロセッサやメモリの周波数を下げると処理能力は下がるが、同時に消費電力も大きく削減される。
そのため、性能のクリティカルパス上にないモジュールの動作周波数を下げることにより、
システム全体としての性能低下を防ぎつつ消費電力を下げることができる。
また、それによって生じた余剰電力をクリティカルパス上のモジュールに融通することで、
電力制約下での性能向上を実現できる。


%というものであり、この技術をHPC領域に応用することによる、電力制約下での性能向上が期待されている。

一方で、蓄電池のような電力資源を利用した電力制約下での性能向上手法も考えられる。
現在のデータセンタやスーパーコンピュータなどの大規模高性能計算システムにおいては、BCM(事業継続マネジメント)の観点から、
地震や火事などの災害による停電時に備えて、自家発電設備や蓄電池が搭載されているケースが多くなってきた。
ただし、それらの設備はあくまで緊急時のための予備電源という扱いであり、平常時に使用されることはない。
そこで、平常時においては、電力要求が高い場面において蓄電池から電力を供給することができ、電力制約下での性能向上に役立てることができる。
この方法をデータセンタに適用する例は文献\cite{Govindan:2011:BLT:2024723.2000105}に示されている。
しかし、スーパーコンピュータなどの高性能計算システムにおいて適用された例は未だ存在しない。
データセンタと高性能計算システムでは、動作アプリケーションやハードウェア構成も大きく異なるため、
必要となる制御手法も大きく異なると考えられる。


そこで本論文では、高性能計算システムを対象とし、動作周波数と蓄電池の充放電をアプリケーションの動作状況に応じて適切に制御することで、
電力制約下で性能を向上させる手法を提案する。
本手法では、Tapasya Patkiらの研究\cite{Patki:2013:EHO:2464996.2465009}の対象となっているような、
厳しい電力制約のために全てのモジュールを常に最高動作速度で動作させることはできないようなシステムを対象とする。
具体的な手法として、まずアプリケーションのテスト実行時のプロファイルデータからアプリケーションの電力−実行時間曲線の時間推移を予測する。
そして、消費電力を減らしても性能が下がりにくい部分を見つけて充電し、逆に消費電力を増やすと大きく性能が上がる部分で放電する。


以降、2章では本論文に関する技術や研究を紹介し、3章では解くべき問題の定義と、提案手法の核となる論理を説明する。4章では3章での手法の有用性を確認するための実験方法について説明した後、実験結果を示し、それについての考察を述べる。5章では結論と今後の課題を述べる。


%\section{背景}
%\label{sec:background}


%\section{目的}


%\section{本論文の構成}

