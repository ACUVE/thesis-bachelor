\chapter{序論}

現代社会においてコンピュータの担う役割はかつてないほど大きくなっており、コンピュータシステムへのますますの高性能化が強く要求されている。より高性能なコンピュータを作るべく、これまで多くの研究者がコンピュータ技術の発展に貢献し、Mooreの法則\cite{mooreslaw}の示す通りチップの集積度が指数関数的に増すと共にコンピュータの性能も向上し続けている。

近年、コンピュータの性能向上の妨げとなっている要因の一つが消費電力の増大である。一般に、高速な演算を行うためには大きな電力を消費しなければならず、数年前までは性能向上と共に消費電力も増加し続けてきた。ところがスーパーコンピュータなどのHPC領域においては既に供給できる限界に近い電力を消費しており、物理的な電力供給能力によってコンピュータの性能が制限されている。そのため、与えられた電力制約の中でいかに処理能力を向上させるかが現在の大きな課題となっている。

この課題を解決するため、プロセッサやメモリの動作速度を動的に制御するDVFSという技術が開発され、現在の多くのコンピュータに搭載されている。一般に、プロセッサやメモリの周波数を下げると処理能力は下がるが、同時に消費電力も大きく削減される。DVFSを用いた電力削減技術とは、性能のクリティカルパス上にないモジュールの動作速度を落とすことにより、性能低下を防ぎつつ消費電力を下げるというものであり、この技術をHPC領域に応用することによる、電力制約下での性能向上が期待されている。

一方で最近の研究では、蓄電池のような今まで電力資源を利用した電力制約下での性能向上手法も提案されている\cite{Govindan:2011:BLT:2024723.2000105}。現在のデータセンターやスーパーコンピュータなどの大規模高性能計算システムにおいては、BCM(事業継続マネジメント)の観点から、地震や火事などの災害による停電時にも継続してコンピュータを稼働させられるように自家発電設備や蓄電池が搭載されているケースが多くなってきた。ただ、現状ではそれらの設備はあくまで緊急時のための予備電源としてのみ見なされており、平常時には使用されていない。そこで、電力要求が高い部分では蓄電池から電力を供給することにより電力制約を守りつつ性能低下を防ぐことができるなど、これらの電力資源活用による電力制約下での性能向上の可能性が示唆されている。しかしまだこの可能性が示唆されてから日が浅く、未開拓の領域が多く残されている。

そこで本論文では蓄電池が搭載された高性能計算システムにおいて、DVFSを用いて電力制約を守りながらも積極的に蓄電池の充放電を行うことによって、電力制約下での性能向上手法を提案する。HPC領域において蓄電池を用いた電力制約下における性能向上手法はいまだ提案されておらず、本稿において初めての試みである。本手法では、Tapasya Patkiらの研究\cite{Patki:2013:EHO:2464996.2465009}の対象となっているような、厳しい電力制約のために全てのモジュールを常に最高動作速度で動作させることはできないようなシステムを対象とする。まずアプリケーションのテスト実行時のプロファイルデータからアプリケーションの電力対性能グラフの時間推移を予測する。そして消費電力を減らしても性能が下がりにくい部分を見つけて充電し、逆に消費電力を増やすと大きく性能が上がる部分で放電することにより、電力制約下における性能向上を目指す。

以降、2章では本論文に関する技術や研究を紹介し、3章では解くべき問題の定義と、提案手法の核となる論理を説明する。4章では3章での手法の有用性を確認するための実験方法について述べる。5章で実験結果を示し、6章でその結果について考察した後、7章で結論と今後の課題を述べる。


%\section{背景}
%\label{sec:background}


%\section{目的}


%\section{本論文の構成}

